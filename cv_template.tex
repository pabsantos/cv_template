\documentclass[12pt, a4paper]{article}

\usepackage{titlesec}
\usepackage[margin=20mm]{geometry}
\usepackage{multicol}
\usepackage[colorlinks=true, urlcolor=blue]{hyperref}
\usepackage{titling}
\usepackage{xcolor}
\usepackage[T1]{fontenc}


\definecolor{nordblue}{HTML}{155259}

\setlength{\parindent}{0em} % No paragraph indentation

\titleformat{\section}
{\color{nordblue}\normalsize\bfseries\uppercase}
{}
{0em}
{}[\titlerule]

\titleformat{\subsection}
{\color{nordblue}\normalsize\bfseries}
{}
{0em}
{}

\titleformat{\subsubsection}[runin]
{\color{nordblue}\bfseries}
{}
{0em}
{}[:]


\titlespacing{\section}
{0em}{20pt}{10pt}

\titlespacing{\subsection}
{0em}{15pt}{5pt}

\renewcommand{\maketitle}{
   \begin{center}
      \large
      \textbf{\theauthor}
      \small
      \vspace{5pt}

      \href{mailto:pedroaugusto@ufpr.br}{pedroaugusto@ufpr.br} --- (41) 99893 0278 --- Curitiba-PR
      \par \href{https://www.linkedin.com/in/pabsantos/}{LinkedIn} --- \href{https://github.com/pabsantos}{GitHub} --- \href{https://www.researchgate.net/profile/Pedro-Santos-145}{ResearchGate}
   \end{center}
   }

\author{Pedro Augusto Borges dos Santos}

\begin{document}

\maketitle

\section{Educação}

\subsection{Mestrado em Planejamento Urbano}

\textbf{Instituição:} Universidade Federal do Paraná; Programa de Pós-graduação em Planejamento Urbano.

\textbf{Período:} Mar/2020 - Mar/2022 (previsão).

\textbf{Título:} \textit{The Impact of the Built Environment on Speeding Behavior in Curitiba - Brazil}.

\subsection{Graduação em Engenharia Civil}

\textbf{Instituição:} Universidade Federal do Paraná.

\textbf{Período:} Mar/2013 - Jul/2019.

\textbf{Título:} \textit{Microssimulação como ferramenta de análise de desempenho de interseções: Estudo de caso em Curitiba}.

\section{Experiência}

\subsection{Residência Técnica}

\textbf{Instituição:} Observatório Nacional de Segurança Viária.

\textbf{Período:} Jan/2021 - Atual.

\textbf{Projeto:} \textit{Análise estratégica da segurança viária no Brasil: um estudo para implementação do Plano Nacional de Redução de Mortes e Lesões no Trânsito}.

\subsection{Analista de Pesquisa}

\textbf{Instituição:} Observatório Nacional de Segurança Viária.

\textbf{Período:} Mar/2020 - Dez/2020.

\textbf{Atividades:} Análise de dados e produção de relatórios relacionados à área da segurança viária no Brasil.

\subsection{Estagiário}

\textbf{Instituição:} Observatório Nacional de Segurança Viária.

\textbf{Período:} Ago/2018 - Ago/2019.

\textbf{Atividades:} Coleta, processamento e análise de dados relacionados à área da segurança viária no Brasil.

\subsection{Estagiário}

\textbf{Instituição:} Vettor Engenharia de Projetos.

\textbf{Período:} Set/2017 - Mar/2018.

\textbf{Atividades:} Desenvolvimento de projetos hidrossanitários.

\section{Projetos de pesquisa}

\subsection{Estudo Naturalístico de Direção Brasileiro}

\textbf{Instituição:} Centro de Estudos em Planejamento e Políticas Urbanas (CEPPUR-UFPR)

\textbf{Período:} Dez/2019 - Atual

\textbf{Coordenador:} Dr. Jorge Tiago Bastos

\textbf{Financiamento:} CNPq; Observatório Nacional de Segurança Viária; Mobi7

%\section{Produção acadêmica}

%\subsection{Artigos publicados em periódicos}

%BASTOS, JORGE TIAGO ; \textbf{DOS SANTOS, PEDRO AUGUSTO B.} ; AMANCIO, EDUARDO CESAR ; GADDA, TATIANA MARIA C. ; RAMALHO, JOSÉ AURÉLIO ; KING, MARK J. ; OVIEDO-TRESPALACIOS, OSCAR . Is organized carpooling safer? Speeding and distracted driving behaviors from a naturalistic driving study in Brazil. ACCIDENT ANALYSIS AND PREVENTION, v. 152, p. 105992, 2021. \\

%BASTOS, JORGE TIAGO ; \textbf{SANTOS, PEDRO AUGUSTO B. DOS} ; AMANCIO, EDUARDO CESAR ; GADDA, TATIANA MARIA C. ; RAMALHO, JOSÉ AURÉLIO ; KING, MARK J. ; OVIEDO-TRESPALACIOS, OSCAR . Naturalistic Driving Study in Brazil: An Analysis of Mobile Phone Use Behavior while Driving. International Journal of Environmental Research and Public Health, v. 17, p. 6412, 2020. 

%\subsection{Artigos publicados anais de congressos}

%SUGUINOSHITA, M. C. ; VALEIXO, G. R. R. ; \textbf{SANTOS, P. A. B.} ; BASTOS, JORGE TIAGO . Fatores Determinantes para o Excesso de Velocidade em Vias Arterias Urbanas. In: 2º Simpósio de Transportes do Paraná, 3º Seminários em Aeroportos e Transporte Aéreo e 3º Urbanidade., 2020, Curitiba. Livro de Resumos e Trabalhos Completos do Simpósio de Transportes do Paraná (STPR), 2020. p. 147-158. 

%\subsection{Comunicações Técnicas}

%BORGUEZANI, J. R. ; \textbf{SANTOS, P. A. B.} ; OSORIO, F. S. ; BASTOS, J. T. . Plataforma de coleta de dados naturalísticos de segurança viária. In: 34º Congresso de Pesquisa e Ensino em Transporte da ANPET, 2020. Anais do 34º Congresso de Pesquisa e Ensino em Transportes, 2020. p. 2610-2617. \\

%\textbf{SANTOS, P. A. B.}; BASTOS, J. T. ; GARONCE, F. V. . Avaliação do Conteúdo dos Portais dos Departamentos Estaduais de Trânsito - DETRANs. In: 2º Simpósio de Transportes do Paraná, 3º Seminários em Aeroportos e Transporte Aéreo e 3º Urbanidade., 2020, Curitiba. Livro de Resumos e Trabalhos Completos do Simpósio de Transportes do Paraná (STPR), 2020. v. 1. p. 121-128. 

%\subsection{Livros}

%BASTOS, J. T. ; GARONCE, F. V. ; \textbf{SANTOS, P. A. B. }; IGARASHI, A. V. ; ANDRADE, G. A. M. . Desempenho brasileiro na década de ação pela segurança no trânsito: análise, perspectivas e indicadores 2011-2020. 1. ed. Brasília: Viva Editora, 2020. v. 1. 120p . 

%\subsection{Resumos}

%RAMALHO, JOSÉ AURÉLIO ; MEGID JUNIOR, J. ; GARONCE, F. V. ; BASTOS, JORGE TIAGO ; \textbf{SANTOS, P. A. B.} ; IGARASHI, A. V. ; RODRIGUES, F. ; OBELHEIRO, M. R. . Forgiving roadways: the Brazilian case.. In: 2nd Smart Transport Infrastructures Summit (STIS 2021), 2021, Dar es Salaam. Anais do 2nd Smart Transport Infrastructures Summit (STIS 2021), 2021.  \\

%RAMALHO, J. A. ; MEGID JUNIOR, J. ; BASTOS, J. T. ; \textbf{SANTOS, P. A. B.} ; IGARASHI, A. V. . Road safety management tool: Observation, Monitoring and Action System (SOMA). In: 2nd Smart Transport Infrastructures Summit (STIS 2021), 2021, Dar es Salaam. Anais do 2nd Smart Transport Infrastructures Summit (STIS 2021), 2021. \\

%BASTOS, JORGE TIAGO ; \textbf{SANTOS, P. A. B.} ; SZELIGA, R. A. ; IGARASHI, A. V. ; CHAVES, B. H. S. ; TRES, G. ; RAMALHO, JOSÉ AURÉLIO ; ROMAO, M. N. P. V. ; FERRAZ, A. C. P. . Brazilian Naturalistic Driving Study (NDS-BR). In: 2nd Smart Transport Infrastructures Summit (STIS 2021), 2021, Dar es Salaam. Anais do 2nd Smart Transport Infrastructures Summit (STIS 2021), 2021

\section{Habilidades}

\subsection{Programação em R:}

\begin{itemize}
   \item Análise, manipulação e visualização de dados com auxílio dos pacotes do \verb|tidyverse|;
   \item Criação e análise de modelos estatísticos;
   \item Análise, manipulação e visualização de dados espaciais com auxílio dos pacotes \verb|sf|, \verb|rgeos|, \verb|raster|, \verb|RQGIS|, \verb|tmap|, \verb|ggplot|, \verb|leaflet| e \verb|osmdata|;
   \item Criação de relatórios e apresentações no R Markdown. 
   \item Criação de apps e dashboards com auxílio do pacote \verb|shiny|
\end{itemize}

\subsection{QGIS:}

\begin{itemize}
   \item Criação de mapas;
   \item Manipulação de dados espaciais.
\end{itemize}

\subsubsection{\LaTeX}

Criação de relatórios, artigos e apresentações.

\subsubsection{Python}

Manipulação básica de dados com auxílio do pacote \verb|pandas|, \verb|openpyxl| e \verb|geopandas|.

%\subsubsection{SQL}

%Execução de queries para manipualção básica em banco de dados.

%\subsubsection{AutoCAD}

%Detalhamento em projetos completentares de construção civil.

\subsubsection{Pacote Office:}

Experiência avançada em Word, Excel e Powerpoint.



%\section{Referências acadêmicas}
%Dr. Jorge Tiago Bastos

%Contato: \href{mailto:jtbastos@ufpr.br}{jtbastos@ufpr.br}

\end{document}